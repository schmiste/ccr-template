\documentclass{sig-alternate-05-2015}


\usepackage{balance}

\setcopyright{none}


\makeatletter
\def\@copyrightspace{\relax}
\makeatother


\usepackage{fancyhdr}
\fancypagestyle{plain}{%
   \fancyhf{} %
   \renewcommand{\headrulewidth}{0pt}%
   \fancyfoot[L]{ACM SIGCOMM Computer Communication Review }%
   \fancyfoot[R]{Volume 47 Issue 1, January 2017}%
}
\pagestyle{plain}
\def\refname{}

\begin{document}


\title{CCR Example File}

\numberofauthors{5}
\author{
\begin{tabular*}{0.99\textwidth}%
{@{\extracolsep{\fill}}ccc}
A & B & C\\
\affaddr{TU Darmstadt, Germany} & \affaddr{Northeastern University, USA}
&        \affaddr{KTH, Sweden}\\ 
\email{a@tu-darmstadt.de} & \email{b@neu.edu}
& \email{c@kth.se}
\end{tabular*}  \vspace{3mm} \\
\begin{tabular*}{0.99\textwidth}%
{@{\extracolsep{\fill}}cc}
D & E\\
       \affaddr{ETH, Switzerland} & \affaddr{Aalborg University, Denmark}\\
       \email{d@inf.ethz.ch} & \email{e@cs.aau.dk}
\end{tabular*}\\
\begin{tabular}{c}
\end{tabular}\\
\begin{tabular}{c}
{\normalsize This article is an editorial note submitted to CCR. It has NOT been peer reviewed.}\\
{\normalsize The authors take full responsibility for this article's
technical content. Comments can be posted through CCR Online.}
\end{tabular}
}

\maketitle

\begin{abstract}
Based on the Dagstuhl Seminar 15102,
this paper initiates the study of more structured approaches to
describe secure routing protocols and the corresponding attacker models,
in an effort to better understand existing secure routing protocols,
and to provide a framework for designing new protocols.
\end{abstract}

\begin{CCSXML}
<ccs2012>
<concept>
<concept_id>10003033.10003039.10003045.10003046</concept_id>
<concept_desc>Networks~Routing protocols</concept_desc>
<concept_significance>500</concept_significance>
</concept>
</ccs2012>
\end{CCSXML}

\begin{CCSXML}
<ccs2012>
<concept>
<concept_id>10002978.10003014.10003015</concept_id>
<concept_desc>Security and privacy~Security protocols</concept_desc>
<concept_significance>500</concept_significance>
</concept>
</ccs2012>
\end{CCSXML}

\ccsdesc[500]{Networks~Routing protocols}
\ccsdesc[500]{Security and privacy~Security protocols}

\printccsdesc

\keywords{Taxonomy, Adversarial Models}


\section{Introduction}\label{sec:intro}

Communication networks have become a critical infrastructure, as other critical
infrastructures increasingly rely on them.
As routing lies at the heart of 
any communication network,
the security of the underlying 
routing protocol is crucial
to prevent attacks and ensure availability.
However, the routing system is not only one
of the most complex and fragile components in the
global information infrastructure, but also 
one of the least protected ones~\cite{route-infra}.


\noindent \textbf{Acknowledgments.} The discussions
leading to this editorial were initiated during
Dagstuhl Seminar 15102 on 
\emph{Secure Routing for Future Communication Networks},
and we thank all participants for their contributions. 

{ \balance
{
%\bibliographystyle{abbrv}
%\bibliography{literature}
\begin{thebibliography}{1}

\bibitem{route-infra}
D.~Montgomery and S.~Murphy.
\newblock Toward secure routing infrastructures.
\newblock {\em Security Privacy, IEEE}, 4(5):84--87, 2006.

\end{thebibliography}
}
}

\end{document}

