\documentclass[sigconf]{acmart}

% Disable / remove copyright boxes
\setcopyright{none}
\settopmatter{printacmref=false}
\renewcommand\footnotetextcopyrightpermission[1]{}

% Increase margin between text and footer
\setlength{\footskip}{20pt}

% Add CCR footer
\usepackage{fancyhdr}
\fancypagestyle{plain}{%
   \fancyhf{} %
   \fancyfoot[L]{ACM SIGCOMM Computer Communication Review}%
   \fancyfoot[R]{Volume 48 Issue 1, January 2018}%
}
\pagestyle{plain}

% Add CCR footer on first page
\fancypagestyle{firstpagestyle}{%
   \fancyhf{} %
   \fancyfoot[L]{ACM SIGCOMM Computer Communication Review}%
   \fancyfoot[R]{Volume 48 Issue 1, January 2018}%
}

% Add editorial note
\begin{teaserfigure}
	\parbox{\textwidth}{\centering\normalsize
		This article is an editorial note submitted to CCR. It has NOT been peer reviewed.\\
		The authors take full responsibility for this article's
		technical content. Comments can be posted through CCR Online.
	}
	\vspace{10pt}
\end{teaserfigure}

\usepackage{balance}

\begin{document}

\title{CCR Example File}

\author{A}
\affiliation{
	\institution{TU Darmstadt, Germany}
}
\email{a@tu-darmstadt.de}

\author{B}
\affiliation{
	\institution{Northeastern University, USA}
}
\email{b@neu.edu}

\author{C}
\affiliation{
	\institution{KTH, Sweden}
}
\email{c@kth.se}

\author{D}
\affiliation{
	\institution{ETH, Switzerland}
}
\email{d@inf.ethz.ch}

\author{E}
\affiliation{
	\institution{Aalborg University, Denmark}
}
\email{e@cs.aau.dk}

\begin{abstract}
	Based on the Dagstuhl Seminar 15102,
	this paper initiates the study of more structured approaches to
	describe secure routing protocols and the corresponding attacker models,
	in an effort to better understand existing secure routing protocols,
	and to provide a framework for designing new protocols.
\end{abstract}

\begin{CCSXML}
	<ccs2012>
	<concept>
	<concept_id>10003033.10003039.10003045.10003046</concept_id>
	<concept_desc>Networks~Routing protocols</concept_desc>
	<concept_significance>500</concept_significance>
	</concept>
	</ccs2012>
\end{CCSXML}

\begin{CCSXML}
	<ccs2012>
	<concept>
	<concept_id>10002978.10003014.10003015</concept_id>
	<concept_desc>Security and privacy~Security protocols</concept_desc>
	<concept_significance>500</concept_significance>
	</concept>
	</ccs2012>
\end{CCSXML}

\ccsdesc[500]{Networks~Routing protocols}
\ccsdesc[500]{Security and privacy~Security protocols}

\keywords{Taxonomy, Adversarial Models}

\maketitle

\section{Introduction}\label{sec:intro}

Communication networks have become a critical infrastructure, as other critical
infrastructures increasingly rely on them.
As routing lies at the heart of
any communication network,
the security of the underlying
routing protocol is crucial
to prevent attacks and ensure availability.
However, the routing system is not only one
of the most complex and fragile components in the
global information infrastructure, but also
one of the least protected ones~\cite{route-infra}.

\begin{acks}
	The discussions leading to this editorial were initiated during
	Dagstuhl Seminar 15102 on
	\emph{Secure Routing for Future Communication Networks},
	and we thank all participants for their contributions.
\end{acks}

{ \balance
{
	%\bibliographystyle{ACM-Reference-Format}
	%\bibliography{literature}
	\begin{thebibliography}{1}

		\bibitem{route-infra}
		D.~Montgomery and S.~Murphy.
		\newblock Toward secure routing infrastructures.
		\newblock {\em Security Privacy, IEEE}, 4(5):84--87, 2006.

	\end{thebibliography}
}
}

\end{document}
